\chapter{PENDAHULUAN}

\section{Latar Belakang}
Pendidikan merupakan aspek utama bagi suatu Negara dalam meningkatkan daya saing padmasa mendatang. Salah satu penunjang keberhasilan dan kelancaran pendidikan adalah adanya fasilitas pendidikan. Menurut Undang-Undang Sistem Pendidikan NasionalNomor 20 Tahun 2003 Pasal 45 Ayat 1 Menyatakan: “Setiap satuan pendidikan formal dan nonformal menyediakan sarana dan prasarana yang memenuhi keperluan pendidikan sesuai dengan pertumbuhan perkembangan  potensi  fisik,  kecerdasan  intelektual,  sosial, emosional dan kejiwaan peserta didik". Sarana merupakan alat yang langsung untuk mencapai tujuan pendidikan, seperti buku, perpustakaan, dan lain sebagainya. Sedangkan prasarana merupakan alat yang tidak langsung untuk mencapai tujuan pendidikan, seperti bangunan sekolah, lapangan olahraga, dan sebagainya.

Berdasarkan data Badan Pusat Statistika Kabupaten Mojokerto, Jumlah sekolah Jenjang SMP tahun 2021/2022 mengalami penurunan. Peristiwa ini tentunya  berpengaruh  pada  pembangunan  Pendidikan  di  Kabupaten Mojokerto. Ketidakmerataan pendidikan di Kabupaten Mojokerto disebabkan oleh timpangnya layanan Pendidikan dan ketersediaannya fasilitas pendidikan masing-masing kecamatan. Selain itu juga disebabkan oleh minimnya kesadaran akan pentingnya pendidikan. Hal ini tentunya berpengaruh pada tingkat sumber daya manusia dan dapat meningkatkan angka kemiskinan di Kabupaten Mojokerto. 

Praktik Kerja Lapangan merupakan sebuah wadah bagi mahasiswa untuk mengaplikasikan ilmu yang sudah diperoleh ketika proses pembelajaran di bangku perkuliahan ke dalam dunia kerja. Selain itu, melalui Praktik Kerja Lapangan mahasiswa akan memperoleh kesempatan untuk mengembangkan cara berpikir secara kritis dan logisPada dunia Pendidikan hubungan antara teori dan praktik merupakan hal yang penting agar dapat membandingkan dan membuktikan  ilmu  yang  sudah  dipelajari  dengan  keadaan  di  lapangan. Mengingat saat ini makinbanyaknya angka pengangguran terdidik dan makinketatnya persaingan dalam dunia kerja, sehingga mengharuskan mahasiswa untuk mempersiapkan diri sejak dini. Softskill yang dimiliki mahasiswa dapat membantu dalam bersaing dan menjadi tenaga kerja yang kompeten di bidangnya.

Pada kegiatan Praktik Kerja Lapangan, kami memilih Badan Pusat Statistik Kabupaten Mojokerto. Sebab, Badan Pusat Statistik Kabupaten Mojokerto merupakan salah satu instansi yang erat kaitannya dengan matematika.Pada dasarnya matematika memiliki peranan penting seluruh instansi, baik dalam hal analisis data, pengolahan data, pengelompokan data, dan sebagainya sehingga melalui kegiatan Praktik  Kerja  Lapanganini  kami  mampu menemukan suatu permasalahan yang kemudian dapat menganalisis danmenemukan keputusan dari permasalahan tersebut.Selain itu kami dapat turut serta berperan dalam mencapai visi misi Badan Pusat Statistik Kabupaten Mojokertosebagai penyedia data statistikyang berkualitas. 

Terdapat  beberapa  penelitian  terdahulu  yang  erat  kaitannya  dengan penelitian  ini,  diantaranya  mengenai pengelompokan  tingkat  pendidikan berdasarkan jumlah sekolah Di Provinsi Jawa Barat menggunakan algoritma K-Means. Pada penelitian ini diperoleh hasil Kabupaten/Kota di wilayah Provinsi Jawa Barat yang telah melewati proses clustering menggunakan metode  K-Means  dibagi  menjadi  tiga  cluster  indeks  pendidikan  yang dikategorikan menjadi tiga wilayah, yaitu wilayah yang memiliki tingkat pendidikan tinggi, wilayah yang memiliki tingkat pendidikan sedang/menengah, dan wilayah yang memiliki tingkat pendidikan rendah (Alfianti,2022).Selain itu, terdapat penelitian mengenai data mining algoritma K-Means dalam mengelompokkan jumlah desa yang memiliki fasilitas sekolah menurut Provinsi berdasarkan tingkat pendidikan. Penelitian ini bertujuan untuk mengetahui provinsi mana saja yang memiliki jumlah desa yang memiliki fasilitas sekolah menurut provinsi berdasarkan tingkat Pendidikan. Di mana hasil yang diperoleh dari penelitian ini adalah terbentuk tiga cluster jumlah desa yang memiliki fasilitas sekolah menurut provinsi berdasarkan tingkat pendidikan yaitu cluster satu, cluster dua dengan dan cluster tiga (Eryzha,dkk., 2018). Serta terdapat penelitian mengenaioptimasi cluster K-Means menggunakan metode Elbow pada data pengguna narkoba dengan pemrograman Python. Pada penelitian ini, diperoleh hasil bahwa melalui eksekusi Python, informasi yang dihasilkan sesuai dengan keadaan yang ada. Namun pada beberapa cluster terdapat data yang dianggap tidak berada pada cluster yang tepat. Sebab, penentuan titik pusat cluster pengujian awal yang ditentukan sendiri oleh Python dan tidak dapat diubah sehingga memengaruhi pula perhitungan yang ada (Winarta, 2021).

Berdasarkan pemaparan sebelumnyadan adanya publikasitahunan Badan Pusat Statistika Kabupaten Mojokerto tahun 2022, yaituKecamatan Dalam Angka 2022 yang salah satunya berisi informasi mengenai pendidikan. Dimana pada publikasi tersebut terdapat fenomena penurunan jumlah sekolah Jenjang SMP tahun 2021/2022. Oleh karena itu, perlu adanya pengelompokan kecamatan di Kabupaten Mojokerto berdasarkan jumlah sekolah di Kabupaten Mojokerto pada tahun 2021/2022. Salah satu algoritma yang dapat digunakan pada pengelompokan ini adalahmenggunakan algoritma K-Means dengan menggunakan metode jarak Euclidean dan menggunakan Software Python. Diharapkan hasil dari pengelompokkan ini dapat menjadi masukan bagi pemerintah dalam meningkatkan fasilitas sekolah, terutama pada kecamatan yang berada kelompok tingkat rendah.

\section{Tujuan}
Tujuan yang ingin dicapai dalam pelaksanaan Praktik Kerja Lapanganadalah sebagai berikut: 
\begin{daftar}
	\item Memperoleh pengalaman dan perluasan ilmu-ilmu di tempat Praktik Kerja Lapangan yang belum diperoleh oleh mahasiswa dibangku perkuliahan.
	\item Meningkatkan pengetahuan dan wawasan mahasiswa khususnya dibidang analisis datadan statistika. 
	\item Meningkatkan  keterampilan  mahasiswa  yang  diperoleh  dibangku perkuliahan.
	\item Meningkatkan rasa tanggung jawab dan sikap profesional pada mahasiswa untuk menambah pengalaman dalam persiapan terjun kedunia kerja.
\end{daftar}

\section{Manfaat}
Manfaat  yang  diharapkan  dalam  pelaksanaan Praktik  Kerja  Lapangan adalah sebagai berikut
\begin{daftar}
	\item Bagi Instansi
	\begin{subdaftar}
		\item Dapat  membantu  meringankan  pekerjaan  para  pegawai  di Badan Pusat Statistik Kabupaten Mojokerto.
		\item Dapat memberikan  masukan dan pertimbangan untuk meningkatkan kualitas dan kuantitas Badan Pusat Statistik Kabupaten Mojokerto.
	\end{subdaftar}
	\item Bagi Mahasiswa
	\begin{subdaftar}
		\item Dapat mempraktikkan  ilmu  yang  didapatkan  di  bangku  perkuliahan secara langsung dalam dunia kerja.
		\item Dapat melatih diri dan menambah pengalaman dalam dunia kerja.
		\item Dapat  meningkatkan  kemampuan  dalam  menangani  permasalahan analisis  datadan  statistikadan  mencari  solusi  atas  permasalahan tersebut.
	\end{subdaftar}
	\item Bagi Universitas
	\begin{subdaftar}
		\item Dapat memberikan tambahan referensi Praktik Kerja Lapanganbagi mahasiswa selanjutnya di bidang matematika.
		\item Dapat  mempererat  kerjasama  antara  Program  Studi  S1  Matematika Jurusan  Matematika Fakultas Matematika  dan  Ilmu  Pengetahuan Alam Universitas   Negeri   Surabaya dan Badan   Pusat   Statistik Kabupaten Mojokerto.
	\end{subdaftar}
\end{daftar}